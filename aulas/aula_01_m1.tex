\documentclass[aspectratio=169]{beamer}

\usetheme{metropolis} % Você pode escolher um tema diferente, como metropolis, etc.
% default - Tema padrão.
% metropolis - Um tema moderno e popular.
% Madrid - Um tema clássico.
% Berlin - Outro tema clássico.
% Copenhagen - Tema com um design limpo.
% Warsaw - Tema com um design simples.
% AnnArbor - Tema com cores vibrantes.
% Bergen - Tema com um design elegante.
% PaloAlto - Tema com um design minimalista.
% Rochester - Tema com um design formal.

\usepackage[utf8]{inputenc} % Codificação UTF-8 (geralmente necessária)
\usepackage[T1]{fontenc}    % Codificação de fontes
\usepackage[brazil]{babel}   % Para hifenização e configurações em português
\usepackage{graphicx}       % Para incluir imagens
\usepackage{amsmath}        % Para equações
\usepackage{hyperref}       % Para links clicáveis (URLs, referências, etc.)
\usepackage{ragged2e}        % Melhor alinhamento justificado
\usepackage{booktabs}       % Para tabelas mais bonitas (toprule, midrule, bottomrule)
\usepackage{fontawesome}  %Para icones como o github
\usepackage{tikz}
\usetikzlibrary{decorations.text} %For text along path
\usepackage[many]{tcolorbox}
\newcommand{\hsp}{\hspace{20pt}}
\newtcbox{\mybox}[1][]{nobeforeafter,colframe=gray!75!black,colback=white,
  colupper=black,boxrule=0.5pt,arc=4pt,boxsep=0pt,left=6pt,right=6pt,top=6pt,bottom=6pt,
  hbox,drop shadow southeast,#1}

%Comando pra usar o icone do github
% \newcommand{\github}[1]{%
% \href{#1}{\faGithub}
% }

% \usepackage{minted}         % Para realce de sintaxe (melhor que verbatim)
% % Configurações para o minted (realce de sintaxe)
% \usemintedstyle{monokai} % Escolha um estilo (monokai, colorful, etc.)
% \setminted{
%     fontsize=\footnotesize,  % Tamanho da fonte no código
%     breaklines=true,        % Quebra de linha automática
%     baselinestretch=1.2,  % Espaçamento entre linhas
%     frame=lines,            % Moldura ao redor do código
%     framesep=2mm,          % Espaçamento da moldura
%     tabsize=4               % Tamanho da tabulação
% }


\title{Blockchain 101}
\author{Thiago Nóbrega}
% \author{Thiago Nóbrega \and \github{https://github.com/thiagonobrega}}
\date{} % Data vazia, para não mostrar a data


\begin{document}
    \maketitle
    \section{Casos de Uso}

\begin{frame}
    \frametitle{Casos de uso}
    \begin{columns}[T]
    \begin{column}{0.7\textwidth}
    \begin{itemize}
     \item Geospatial
            \begin{itemize}
                \item [$\rightarrow$] Secure sharing of geospatial wildlife data (SIGMOD) [2]
                \item [$\rightarrow$] Geofences UAV (SIGSPATIAL) [3]
            \end{itemize}
            \item Health System [4-6]
            \begin{itemize}
              \item [$\rightarrow$] Medical records
              \item [$\rightarrow$] Medical Images
            \end{itemize}
            \item Smartcity
    \end{itemize}
    \end{column}
      \begin{column}{0.25\textwidth}
            % \includegraphics[width=\linewidth]{aulas/images/health1.png}
            % \vspace*{0.2cm}
            % \includegraphics[width=\linewidth]{aulas/images/geofence.jpg}
            %  \vspace*{0.2cm}
            % \includegraphics[width=\linewidth]{aulas/images/data.jpg}
            % \vspace*{0.2cm}
            % \includegraphics[width=0.9\linewidth]{aulas/images/smart_city.jpg}
        \end{column}
    \end{columns}
\end{frame}


\begin{frame}
    \begin{center}
           \Large Quais razões levaram à adoção da Blockchain em tantos cenários diferentes?
       \end{center}
\end{frame}
    
\end{document}