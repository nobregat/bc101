\documentclass{beamer}

\usetheme{default} % Você pode escolher um tema diferente, como metropolis, etc.

\usepackage[utf8]{inputenc} % Codificação UTF-8 (geralmente necessária)
\usepackage[T1]{fontenc}    % Codificação de fontes
\usepackage[brazil]{babel}   % Para hifenização e configurações em português
\usepackage{graphicx}       % Para incluir imagens
\usepackage{amsmath}        % Para equações
\usepackage{minted}         % Para realce de sintaxe (melhor que verbatim)
\usepackage{hyperref}       % Para links clicáveis (URLs, referências, etc.)
\usepackage{ragged2e}        % Melhor alinhamento justificado
\usepackage{booktabs}       % Para tabelas mais bonitas (toprule, midrule, bottomrule)
\usepackage{fontawesome}  %Para icones como o github
\usepackage{tikz}
\usetikzlibrary{decorations.text} %For text along path
\usepackage[many]{tcolorbox}
\newcommand{\hsp}{\hspace{20pt}}
\newtcbox{\mybox}[1][]{nobeforeafter,colframe=gray!75!black,colback=white,
  colupper=black,boxrule=0.5pt,arc=4pt,boxsep=0pt,left=6pt,right=6pt,top=6pt,bottom=6pt,
  hbox,drop shadow southeast,#1}

%Comando pra usar o icone do github
\newcommand{\github}[1]{%
\href{#1}{\faGithub}
}

% Configurações para o minted (realce de sintaxe)
\usemintedstyle{monokai} % Escolha um estilo (monokai, colorful, etc.)
\setminted{
    fontsize=\footnotesize,  % Tamanho da fonte no código
    breaklines=true,        % Quebra de linha automática
    baselinestretch=1.2,  % Espaçamento entre linhas
    frame=lines,            % Moldura ao redor do código
    framesep=2mm,          % Espaçamento da moldura
    tabsize=4               % Tamanho da tabulação
}


\title{Blockchain 101}
\author{Thiago Nóbrega \and \github{https://github.com/thiagonobrega}}
\date{} % Data vazia, para não mostrar a data


\begin{document}

\begin{frame}
  \titlepage
  \begin{center}
        \includegraphics[width=0.45\textwidth]{aulas/images/blockchain-7004923_1280.png}
    \end{center}
\end{frame}


% \begin{frame}{Agenda}
%   % Usando TikZ para desenhar o diagrama (mais flexível que mermaid)
%   \begin{tikzpicture}[
%     node distance = 0.8cm,
%     every node/.style = {draw, rounded corners, text width=4cm, align=center},
%     >=stealth, auto
%   ]
%     \node (intro) {Introdução};
%     \node (carac) [below of=intro] {Características};
%     \node (bdvs) [below of=carac] {Blockchain vs Database};

%     \node (appintro) [right=1.5cm of intro] {Introdução};
%     \node (solidity) [below of=appintro] {Solidity};
%     \node (pratica) [below of=solidity] {Prática};

%     \draw[->] (intro) -- (carac);
%     \draw[->] (carac) -- (bdvs);

%     \path[->, decorate, decoration={text along path,text={|\Large|Aplicações Descentralizadas},text align=center}]
%          (appintro.north west) -- (appintro.south east);

%     \draw[->] (appintro) -- (solidity);
%     \draw[->] (solidity) -- (pratica);

%     \node[above=0.5cm of intro, text width=10cm, align=center] {\Large Blockchain 101};

%   \end{tikzpicture}
% \end{frame}


\begin{frame}{Disclaimer}
Informações Importantes.
\end{frame}

\begin{frame}
\frametitle{O que não vamos aprender}

\begin{itemize}
    \item Criptoativos
    \item Valuation
    \item Balanceamento de Carteira
\end{itemize}


\begin{columns}[T] % Alinhamento superior das colunas

    \begin{column}{0.7\textwidth} % Largura da primeira coluna (texto)

    \end{column}
    \begin{column}{0.25\textwidth}
        \includegraphics[width=\linewidth]{aulas/images/brais.png}
        \vspace{0.5cm} % Espaçamento vertical
        \includegraphics[width=\linewidth]{aulas/images/binance.png}
        \vspace{0.5cm}
         \includegraphics[width=\linewidth]{aulas/images/kriptomat.jpg}
    \end{column}
\end{columns}

\end{frame}

\begin{frame}
  \frametitle{Conceitos de Criptografia}
\begin{itemize}
    \item Funções Hash
    \item Assinaturas Digitais
\end{itemize}
\end{frame}

\begin{frame}
\frametitle{Introdução}

\end{frame}

\begin{frame}
    \frametitle{Blockchain}
\end{frame}


\begin{frame}
    \frametitle{TL;DR}
    \begin{center}
       \mybox{\textit{``Blockchain is a imutable append only transaction log''}}
    \end{center}
\end{frame}

\begin{frame}
\frametitle{Casos de uso}


\begin{columns}[T] % Alinhamento superior das colunas

    \begin{column}{0.7\textwidth} % Largura da primeira coluna (texto)
            \begin{itemize}
            \item E-GOV
            \begin{itemize}
                \item [$\rightarrow$] \href{https://repositorio.enap.gov.br/handle/1/4727}{b-CPF}
                \item [$\rightarrow$] \href{https://portal.tcu.gov.br/imprensa/noticias/tcu-e-bndes-lancam-rede-blockchain-brasil-e-definem-proximos-passos.htm}{Assinatura de Contratos}
                \item [$\rightarrow$] \href{https://einvestidor.estadao.com.br/criptomoedas/blockchain-governo-oficializa-uso-para-documentos/}{Validação de documentos}
            \end{itemize}
        \end{itemize}
    \end{column}
    \begin{column}{0.25\textwidth}
        \includegraphics[width=\linewidth]{aulas/images/receita.png}
        \vspace{0.5cm} % Espaçamento vertical
        %\includegraphics[width=\linewidth]{aulas/images/e-gov.jpg} %substitua "e-gov.jpg" pelo nome correto do seu arquivo.
    \end{column}
\end{columns}



\end{frame}



\begin{frame}
\frametitle{Casos de uso}

\begin{columns}[T]
    \begin{column}{0.7\textwidth}
    \begin{itemize}
        \item Financial \& Commercial Service [1]
        \begin{itemize}
            \item [$\rightarrow$] Nasdaq
            \item [$\rightarrow$] \href{https://www.oecd.org/corruption/integrity-forum/academic-papers/Georg\%20Eder-\%20Blockchain\%20-\%20Ghana_verified.pdf}{Honduran land}
            \item [$\rightarrow$] Diamond Track
        \end{itemize}
    \end{itemize}
    \end{column}
    \begin{column}{0.25\textwidth}
        \includegraphics[width=\linewidth]{aulas/images/bitcoin.jpg}\\
        \vspace{0.5cm}
        \includegraphics[width=\linewidth]{aulas/images/el_salvador.jpg}
        \vspace{0.5cm}
        \includegraphics[width=\linewidth]{aulas/images/blood_diamond.png}
    \end{column}
\end{columns}
\end{frame}

\begin{frame}
\frametitle{Casos de uso}
\begin{columns}[T]
\begin{column}{0.7\textwidth}
\begin{itemize}
 \item Geospatial
        \begin{itemize}
            \item [$\rightarrow$] Secure sharing of geospatial wildlife data (SIGMOD) [2]
            \item [$\rightarrow$] Geofences UAV (SIGSPATIAL) [3]
        \end{itemize}
        \item Health System [4-6]
        \begin{itemize}
          \item [$\rightarrow$] Medical records
          \item [$\rightarrow$] Medical Images
        \end{itemize}
        \item Smartcity
\end{itemize}
\end{column}
  \begin{column}{0.25\textwidth}
        \includegraphics[width=\linewidth]{aulas/images/health1.png}
        \vspace*{0.2cm}
        \includegraphics[width=\linewidth]{aulas/images/geofence.jpg}
         \vspace*{0.2cm}
        \includegraphics[width=\linewidth]{aulas/images/data.jpg}
        \vspace*{0.2cm}
        \includegraphics[width=0.9\linewidth]{aulas/images/smart_city.jpg}
    \end{column}
\end{columns}
\end{frame}



\begin{frame}
\frametitle{Curiosidades e Histórico}
\begin{center}
\includegraphics[width=0.7\linewidth]{aulas/images/netflix.png}

\href{https://www.netflix.com/watch/80243756}{https://www.netflix.com/watch/80243756}
\end{center}

\end{frame}

\begin{frame}
 \begin{center}
        \Large Quais razões levaram à adoção da Blockchain em tantos cenários diferentes?
    \end{center}
\end{frame}


\begin{frame}
\frametitle{Características}
\begin{itemize}
    \item Descentralizado
    \item Imutável (\textit{Temper Evident})
    \item Transparente/Auditável
    \item Pode ser utilizado por adversários (\textit{Distrustful Parties})
\end{itemize}
\end{frame}

\begin{frame}
  \frametitle{Características da rede}
\end{frame}

\begin{frame}
\frametitle{Distribuição de dados}
 \begin{center}
        \includegraphics[width=0.8\linewidth]{aulas/images/data_distribution.png}
    \end{center}
\end{frame}

\begin{frame}
\frametitle{Tipos de rede}

\textbf{Permissionada} (permissioned)
\begin{itemize}
    \item Públicas
    \item Controle descentralizado
    \item \textit{i.e.}, criptomoedas
\end{itemize}

\vspace{0.5cm} % Espaço entre as listas

\textbf{Não permissionada} (permissionless)
\begin{itemize}
    \item Privadas (ou consórcio)
    \item Controle centralizado*
    \item \textit{i.e.}, Banco Central, LACChain
\end{itemize}
\end{frame}



\begin{frame}
    \frametitle{Imutável}
\end{frame}

\begin{frame}
    Estrutura da blockchain
    \begin{center}
\begin{tikzpicture}[
    node distance = 1.5cm,
     block/.style = {draw, rectangle, minimum height=1cm, minimum width=1.5cm,align=center} %Estilo dos blocos
  ]

    \node [block] (b1) {block\_1};
    \node [block, right of=b1] (b2) {block\_2};
    \node [block, right of=b2] (b3) {block\_3};
    \node [block, right of=b3] (b4) {block\_4};
     \node [block, right of=b4] (b5) {block\_5};
      \node [block, right of=b5] (b6) {block\_6};

  \path[->]
    (b1) edge (b2)
    (b2) edge (b3)
    (b3) edge (b4)
    (b4) edge (b5)
    (b5) edge (b6)
    ;

\end{tikzpicture}
\end{center}
\end{frame}



\begin{frame}
\frametitle{Funções Hash}
 \begin{columns}[T] % Alinhamento superior das colunas
    \begin{column}{0.65\textwidth}
    \begin{itemize}
    \item Distribuição uniforme
    \item Determinístico
    \item Resistente à colisão
\end{itemize}
\end{column}
\begin{column}{0.3\textwidth}
 \includegraphics[width=\linewidth]{aulas/images/hash.png}
\end{column}
\end{columns}
\end{frame}



\begin{frame}
\frametitle{Hashing}

\[ h(x) = x \mod 2^{256} \]

\vspace{0.5cm}

Funções Hash:

\begin{itemize}
    \item Message Digest (md5)
    \begin{itemize}
        \item 128 bits
    \end{itemize}
    \item Secure Hash Algorithm (SHA)
    \begin{itemize}
        \item 256 bits
    \end{itemize}
\end{itemize}
\href{https://andersbrownworth.com/blockchain/hash}{Exemplo da utilização de Hash}

\end{frame}
\begin{frame}
      Estrutura da blockchain
    \begin{center}
\begin{tikzpicture}[
    node distance = 1.5cm,
     block/.style = {draw, rectangle, minimum height=1cm, minimum width=1.5cm,align=center} %Estilo dos blocos
  ]

    \node [block] (b1) {block\_1};
    \node [block, right of=b1] (b2) {block\_2};
    \node [block, right of=b2] (b3) {block\_3};
    \node [block, right of=b3] (b4) {block\_4};
     \node [block, right of=b4] (b5) {block\_5};
      \node [block, right of=b5] (b6) {block\_6};

  \path[->]
    (b1) edge (b2)
    (b2) edge (b3)
    (b3) edge (b4)
    (b4) edge (b5)
    (b5) edge (b6)
    ;

\end{tikzpicture}
\end{center}
\end{frame}

\begin{frame}
\frametitle{Detalhe do encadeamento}
\begin{center}
\includegraphics[width=0.9\linewidth]{aulas/images/blockchain_detail.png}
\end{center}
\href{https://andersbrownworth.com/blockchain/blockchain}{Exemplo do encadeamento}
\end{frame}


\begin{frame}
    \frametitle{Transparente/Auditável}
    \framesubtitle{Proveniência/procedência do dado}
\end{frame}

\begin{frame}
    \frametitle{Proveniência de dados}
    \begin{enumerate}
        \item O quê?
        \item Quando?
        \item Quem?
    \end{enumerate}

\end{frame}

\begin{frame}
    \begin{center}
         \textbf{Como o blockchain identifica os autores das alterações dos dados?}
    \end{center}
\end{frame}



\begin{frame}
    \frametitle{Criptografia Assimétrica}
\end{frame}

\begin{frame}
\frametitle{Criptografia Assimétrica}
\begin{center}
\textit{
\( M \leftarrow \) mensagem plana \\
\( C \leftarrow \) mensagem codificada \\
\( \text{chave\_pub} \leftarrow \) chave pública \\
\( \text{chave\_priv} \leftarrow \) chave privada \\
\( n \leftarrow \) tamanho da chave em bits
}
\end{center}

\vspace{0.5cm}

\begin{itemize}
    \item \textbf{codifique}(\(M\), \(\text{chave\_pub}\)) = \(M^{\text{chave\_pub}} \mod n\)
    \item \textbf{decodifique}(\(C\), \(\text{chave\_priv}\)) = \(C^{\text{chave\_priv}}\)
\end{itemize}
\end{frame}

\begin{frame}
    \frametitle{Assinatura Digital}
    \framesubtitle{Assinatura}
\begin{center}
 \begin{tikzpicture}[
    node distance = 1cm,
     every node/.style = {align=center},
    >=stealth, auto
  ]

    \node (alice) {Alice};
    \node (bob) [right=3cm of alice] {Bob};
     \node (bc) [right=1.5cm of alice, yshift=-1.5cm] {Blockchain};

    \draw[->] (alice) -- node[above] {1. hash\_doc = sha256(doc)} (alice);
    \draw[->] (alice) -- node[above, yshift=0.2cm] {2. assinatura = codifique(hash\_doc,Alice\_priv\_key)} (alice);
    \draw[->] (alice) -- node[above,xshift=-1cm] {3. envie(assinatura)} (bc);
    \draw[->] (bc) -- node[above,xshift=1cm] {4. documento assinado(alice,doc,assinatura)} (alice);
    \node[below=0.2cm of alice, text width=6cm, align=center,yshift=0.5cm] {Alice assina um documento na Blockchain};

\end{tikzpicture}
\end{center}
\end{frame}

\begin{frame}
  \frametitle{Assinatura Digital}
    \framesubtitle{Validação da Assinatura}
\begin{center}
 \begin{tikzpicture}[
    node distance = 1cm,
     every node/.style = {align=center},
    >=stealth, auto
  ]

    \node (alice) {Alice};
    \node (bob) [right=3cm of alice] {Bob};
     \node (bc) [right=1.5cm of alice, yshift=-1.5cm] {Blockchain};

    \draw[->] (bc) -- node[above,xshift=-1cm] {1. documento assinado(alice,doc,assinatura)} (alice);
    \draw[->] (bob) -- node[above, xshift=-0.8cm] {2. get\_chave\_pub(Alice)} (bc);
      \draw[->] (bob) -- node[above,xshift=-0.8cm] {3. get\_assinatura(doc)} (bc);
    \draw[->] (bc) -- node[above, xshift=1cm] {4. alice\_pk, assinatura} (bob);
 \draw[->] (bob) -- node[above,yshift=0.2cm] {5. hash\_doc = sha256(doc)} (bob);
    \draw[->] (bob) -- node[above,yshift=0.2cm] {6. hash\_doc = decodifique(alice\_pk,assinatura)} (bob);
   

 \node[below=0.2cm of bob, text width=6cm, align=center,yshift=0.5cm] {Como Bob verifica a assinatura de Alice?};

\end{tikzpicture}
\end{center}
\end{frame}


\begin{frame}
    \frametitle{Distrustful Parties}
    \framesubtitle{Algoritmos de consenso}
\end{frame}

\begin{frame}
\begin{center}
\includegraphics[width=0.8\linewidth]{aulas/images/consensus.png}
\end{center}
\end{frame}

\begin{frame}
\frametitle{Algoritmos de consenso das blockchains}
\begin{itemize}
    \item PoW (Proof of Work)
    \begin{itemize}
        \item [$\rightarrow$] Prova de Trabalho
        \item [$\rightarrow$] Miner
        \item [$\rightarrow$] Miners receive block rewards
    \end{itemize}

    \item PoS (Proof of Stake)
        \begin{itemize}
        \item [$\rightarrow$] To become a validator, a coin owner must "stake" a specific amount of coins (\textit{i.e.}, Ethereum 32 ETH)
        \item [$\rightarrow$] Validators
        \item [$\rightarrow$] Validators receive transactions fees as rewards
    \end{itemize}
\item Proof-Of-Authority (PoA)
\end{itemize}
\end{frame}



\begin{frame}
    \frametitle{Limitadores do uso da Blockchain}
\end{frame}

\begin{frame}
    \frametitle{Limitadores do uso da Blockchain}
    \begin{itemize}
        \item Alto custo de armazenamento
        \item Privacidade dos dados
    \end{itemize}
\end{frame}

\begin{frame}
    \frametitle{Blockchain é um tipo de sistema de gerenciamento de dados?}
\end{frame}

\begin{frame}
    \frametitle{Sim}
    ``Untangling blockchain: A data processing view of blockchain systems''
    [8-12]
\end{frame}

\begin{frame}
 \frametitle{Tipos de sistemas}
 \begin{itemize}
     \item \textbf{SGBDs tradicionais}
        \begin{itemize}
            \item [$\rightarrow$] MySQL, Oracle, Postgres, SQLServer
        \end{itemize}
     \item \textbf{SGBDs distribuídos}
     \begin{itemize}
         \item [$\rightarrow$] VoltDB, Oracle*, CosmosDB
     \end{itemize}
    \item \textbf{Sistemas de arquivo distribuídos}
    \begin{itemize}
        \item [$\rightarrow$] HDFS (Apache Hadoop)
    \end{itemize}

 \end{itemize}
\end{frame}

\begin{frame}
\frametitle{Comparativo}
\centering
\begin{tabular}{@{}lccccc@{}}
\toprule
 & \textbf{Controle Centralizado} & \textbf{Transparente} & \textbf{Consultas}  & \textbf{Storage} & \textbf{T. Falha} \\ \midrule
\textbf{BDs tradicionais}   & sim               & não            & complexas & grande         & não            \\
\textbf{BDs distribuídos}  & sim               & não            & complexas & grande        & n.bizantina    \\
\textbf{HDFS}                & sim               & não            & não      & grande          & n.bizantina    \\ \midrule
\textbf{Blockchain}           & \textbf{não}               & \textbf{sim}            & \textbf{chave-valor}       & \textbf{pequena}         & bizantina  \\ \bottomrule
\end{tabular}
\end{frame}

\begin{frame}
\frametitle{Comparativo de transações por segundo}
\begin{center}
    \includegraphics[width=0.8\linewidth]{aulas/images/throughput.png}
\end{center}
\begin{footnotesize}
 [9]
\end{footnotesize}
\end{frame}



\begin{frame}
    \frametitle{Quando não utilizar blockchain}
    \begin{itemize}
        \item Restrições quanto ao número de operações por segundo
        \item Privacidade de armazenamento
        \item Custo de Armazenamento
    \end{itemize}
\end{frame}

\begin{frame}
 \frametitle{Tema do próximo encontro}
 \begin{center}
   Como escrever aplicações descentralizadas utilizando Blockchain
 \end{center}

\end{frame}

\end{document}
